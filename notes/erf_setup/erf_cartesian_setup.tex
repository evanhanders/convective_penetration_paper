\documentclass[12pt,preprint]{article}
\usepackage{natbib}
\bibliographystyle{plain}
\usepackage{latexsym}
\usepackage{graphicx}
\usepackage{epsfig}
\usepackage{amssymb}
\usepackage{amsmath}
\usepackage{epstopdf}

\usepackage{pb-diagram}

\parindent = 0.0 in
\parskip = 0.15 in

\newcommand\Beq{\begin{align}} 
\newcommand\Eeq{\end{align}}

\newcommand\Bfig{\begin{figure}} 
\newcommand\Efig{\end{figure}}

\newcommand\Ra{\mathrm{Ra}}
\newcommand\Pran{\mathrm{Pr}}
\newcommand\Rac{\mathrm{Ra}_{\mathrm{c}}}
\newcommand\Ek{\mathrm{Ek}}
\newcommand\Ro{\mathrm{Ro}}
\newcommand\Nu{\mathrm{Nu}}
\newcommand\Sc{\mathrm{Sc}}

\newcommand\eps{\varepsilon}
\renewcommand\L {\mathcal{L}}

\newcommand{\n}{\\ \nonumber \\ }
\newcommand{\nn}{\nonumber}
\newcommand{\nnn}{\\ \nonumber \\ \nonumber}

\newcommand\ie{\textit{i.e.},~}
\newcommand\eg{\textit{e.g.},~}
\newcommand{\omicron}{o}

\newcommand{\pd}[1]{\partial_{#1}}
\renewcommand{\vec}[1]{\boldsymbol{#1}}
\newcommand{\M}[1]{\mathbf{#1}}
\renewcommand{\dot}{\vec{\cdot}}
\newcommand{\grad}{\vec{\nabla}}
\newcommand{\cross}{\vec{\times}}
\newcommand{\laplacian}{\nabla^2}

\newcommand{\sump}[2]{\sideset{}{'}\sum_{{#1}=0}^{#2}}

\newcommand{\eq}[1]{eq.~(\ref{#1})}
\newcommand{\eqs}[2]{eqs.~(\ref{#1})~\&~(\ref{#2})}
\newcommand{\eqss}[2]{eqs.~(\ref{#1})--(\ref{#2})}

\newcommand{\Eq}[1]{Eq.~(\ref{#1})}
\newcommand{\Eqs}[2]{Eqs.~(\ref{#1})~\&~(\ref{#2})}
\newcommand{\Eqss}[2]{Eqs.~(\ref{#1})--(\ref{#2})}

\newcommand{\fig}[1]{Fig.~(\ref{#1})}
\newcommand{\figs}[2]{Figs.~(\ref{#1})~\&~(\ref{#2})}

\newcommand{\tbl}[1]{Table~(#1)}

\begin{document}

\section{Nondimensional equations}
The nondimensional system of equations we solve is
\begin{align}
&\grad\dot\vec{u} = 0 \\
&\partial_t \vec{u} + \vec{u}\dot\grad\vec{u} = -\grad \varpi + T \hat{z} + \rm{Re}_0^{-1}\grad^2 \vec{u} \\
&\partial_t T + \vec{u}\dot\grad T + w \grad_{\rm{ad}} = \rm{Pe}_0^{-1}\grad^2 T' + \grad\dot[k \grad \overline{T}] + Q
\end{align}
These are just the Boussinesq equations of motion, with some slight tweaks:
\begin{enumerate}
\item There is a nonzero adiabatic temperature gradient, $\partial_z T_{\rm{ad}} = -\grad_{\rm{ad}}$.
We choose this sign convention ($\grad_{\rm{ad}}$ and similarly $\grad_{\rm{rad}}$ will be positive) to align better with the stellar structure community's intuition.
$|\grad T_0|$ and $\grad_{\rm{ad}}$ may be much larger in magnitude than the convective fluctuations.
The background temperature field $T_0$ does not enter into the momentum equation, as it is automatically canceled by a background $\grad \varpi$.
\item Convection is driven by internal heating, $Q$.
I've confirmed with some simple tests that $|u|^2 \approx Q$, and therefore the convective frequency is $f_{\rm{conv}}^2 \approx Q$.
\item The control freefall P\'{e}clet number, $\rm{Pe}_0$, and the diffusivity associated with it, only acts on fluctuations, not on the $k_x = 0$ mode.
For the $k_x = 0$ mode, we model the radiative flux as $F_{\rm{rad}} = -k \grad \overline{T}$, where $k$ is a nonconstant coefficient, $k(z)$.
$k$ may be a larger or smaller diffusivity than $\rm{Pe}_0^{-1}$, and its magnitude is set by $\int Q\, dz \approx -k \grad\overline{T}$.
\end{enumerate}


\section{Model Setup}
We will study a two-layer model in $z = [0, L_z]$ in which the conductivity, $k(z)$ is a step function such that
\begin{equation}
k(z) = \begin{cases}
k_b & z < 1 \\
k_t & z \geq 1
\end{cases}
\end{equation}
where subscript ``t'' refers to the top layer and ``b'' refers to the bottom.
(These won't be true discontinuities, but will be smooth step-like functions achieved with erfs.)

There will be two important input parameters in this system:
\begin{enumerate}
\item The stiffness, 
\begin{equation}
\mathcal{S} = \frac{N_{t}^2}{f_{\mathrm{conv}}^2},
\end{equation}
which is the ratio of the buoyancy frequency (in the stable layer) to the square convective frequency.
\item $\mathcal{P}$, a parameter which determines what the magnitude of the \emph{negative} convective flux would be in an adiabatic penetrative layer with $k = k_t$.
We will define $\mathcal{P} = -F_{\rm{conv, rz}}/F_{\rm{conv}}$, and our expectation is that we get a lot of convective penetration when $\mathcal{P} \ll 1$ and none when $\mathcal{P} \gg 1$.
\end{enumerate}
In addition to these parameters, the magnitude of the heating, $Q$ and the vertical extent of the heating layer, $\delta_H$, must be specified as they set $F_{\rm{conv}} = Q \delta_H$.
We will choose $Q = 1$ and $\delta_H = 0.2$ so that $F_{\rm{conv}} = 0.2$; we will furthermore choose and $F_{\rm{bot}} = \zeta F_{\rm{conv}}$, where $\zeta$ is some arbitrary scalar that sets how much smaller $k$ is in the cz than the RZ; we'll set $\zeta = 10^{-3}$ for now.

Under these choices, the system is determined by five equations,
\begin{enumerate}
\item At some point, the adiabatic gradient is equal to the radiative gradient,
\begin{equation}
k_{\rm{ad}}\grad_{\rm{ad}} = F_{\rm{tot}}.
\end{equation}
\item at $z = 0$, the base of the CZ, the adiabatic temperature gradient is the radiative gradient, $\grad_{\rm{ad}} = \grad_{\rm{rad}}$, and carries $F_{\rm{bot}}$,
\begin{equation}
k_b \grad_{\rm{ad}} = F_{\rm{bot}}
\end{equation}
\item The radiative gradient can carry the flux in the RZ,
\begin{equation}
k_{t} \grad_{\rm{rad,t}} = F_{\rm{tot}}
\end{equation}
\item Assuming $\grad T = -\grad_{\rm{ad}}$ everywhere, the convective flux in the top layer is $F_{\rm{conv,t}} = F_{\rm{conv}} - \Delta k \grad_{\rm{ad}}$, where $F_{\rm{conv}} = Q \delta_H$ and $\Delta k = k_t - k_b$.
Thus, definitionally,
\begin{equation}
\mathcal{P} = -\frac{F_{\rm{conv,t}}}{F_{\rm{conv}}} = \left(\frac{\Delta k \grad_{\rm{ad}}}{Q \delta_H} - 1\right).
\end{equation}
\item Defining $f_{\rm{conv}}^2 \approx Q$ and $N^2 \approx -(\grad_{\rm{rad,t}} - \grad_{\rm{ad}})$, the stiffness gives
\begin{equation}
\mathcal{S} = \frac{N^2}{f_{\mathrm{conv}}^2} = \frac{-(\grad_{\rm{rad,t}} - \grad_{\rm{ad}})}{Q}
= \frac{F_{\mathrm{tot}}}{Q}(k_{\rm{ad}}^{-1} - k_t^{-1})
\end{equation}
\end{enumerate}

Solving this system of equations gives
\begin{align}
k_t = \frac{\delta_H \mathcal{P}}{\mathcal{S}},\qquad
k_b = k_t \frac{\zeta}{1 + \zeta + \mathcal{P}},\qquad
k_{\rm{ad}} = k_t \frac{1 + \zeta}{1 + \zeta + \mathcal{P}},\\
\grad_{\rm{ad}} = \frac{Q \mathcal{S}}{\mathcal{P}}(1 + \zeta + \mathcal{P}),\qquad
\grad_{\rm{rad}} = \frac{Q \mathcal{S}}{\mathcal{P}}(1 + \zeta).
\end{align}

Just to be safe, I'm going to offset the internal heating from the bottom boundary.
We wouldn't need to do this if we didn't have a bottom, impenetrable boundary condition with a viscous boundary layer.
But, since $w \rightarrow 0$ at the bottom boundary, I don't want to inject heating into the domain somewhere where it must be carried by a superadiabatic temperature gradient (I explicitly don't want to study boundary-driven convection).
So
\begin{equation}
Q(z) = \begin{cases}
0 & z < 0.1 \\
Q & 0.1 \leq z < 0.1 + \delta_H, \\
0 & \text{elsewhere}
\end{cases}.
\end{equation}

We then choose $T_0$ to satisfy
\begin{equation}
\partial_z T_0 =
-\begin{cases}
\grad_{\mathrm{ad}} & z \leq 1 \\
\grad_{\mathrm{rad}} & z > 1
\end{cases}
\end{equation}
(again, perhaps not perfectly this, but with smooth transitions).
And we're off to the races.



\bibliography{biblio.bib}
\end{document} 
