\documentclass[12pt]{article}
\usepackage{natbib}
\bibliographystyle{aasjournal}
\usepackage{latexsym}
\usepackage{graphicx}
\usepackage{epsfig}
\usepackage{amssymb}
\usepackage{amsmath}
\usepackage{epstopdf}

\usepackage{pb-diagram}

\parindent = 0.0 in
\parskip = 0.15 in

\newcommand\Beq{\begin{align}} 
\newcommand\Eeq{\end{align}}

\newcommand\Bfig{\begin{figure}} 
\newcommand\Efig{\end{figure}}

\newcommand\Ra{\mathrm{Ra}}
\newcommand\Pran{\mathrm{Pr}}
\newcommand\Rac{\mathrm{Ra}_{\mathrm{c}}}
\newcommand\Ek{\mathrm{Ek}}
\newcommand\Ro{\mathrm{Ro}}
\newcommand\Nu{\mathrm{Nu}}
\newcommand\Sc{\mathrm{Sc}}

\newcommand\eps{\varepsilon}
\renewcommand\L {\mathcal{L}}

\newcommand{\n}{\\ \nonumber \\ }
\newcommand{\nn}{\nonumber}
\newcommand{\nnn}{\\ \nonumber \\ \nonumber}

\newcommand\ie{\textit{i.e.},~}
\newcommand\eg{\textit{e.g.},~}
\newcommand{\omicron}{o}

\newcommand{\pd}[1]{\partial_{#1}}
\renewcommand{\vec}[1]{\boldsymbol{#1}}
\newcommand{\M}[1]{\mathbf{#1}}
\renewcommand{\dot}{\vec{\cdot}}
\newcommand{\grad}{\vec{\nabla}}
\newcommand{\cross}{\vec{\times}}
\newcommand{\laplacian}{\nabla^2}

\newcommand{\sump}[2]{\sideset{}{'}\sum_{{#1}=0}^{#2}}

\newcommand{\eq}[1]{eq.~(\ref{#1})}
\newcommand{\eqs}[2]{eqs.~(\ref{#1})~\&~(\ref{#2})}
\newcommand{\eqss}[2]{eqs.~(\ref{#1})--(\ref{#2})}

\newcommand{\Eq}[1]{Eq.~(\ref{#1})}
\newcommand{\Eqs}[2]{Eqs.~(\ref{#1})~\&~(\ref{#2})}
\newcommand{\Eqss}[2]{Eqs.~(\ref{#1})--(\ref{#2})}

\newcommand{\fig}[1]{Fig.~(\ref{#1})}
\newcommand{\figs}[2]{Figs.~(\ref{#1})~\&~(\ref{#2})}

\newcommand{\tbl}[1]{Table~(#1)}

\newcommand{\aap}{A\&A}

\begin{document}

\section{Dimensional equations}
The dimensional Boussinesq equations are
\begin{align}
&\grad\dot\vec{u} = 0 \label{eqn:continuity} \\
&\partial_t \vec{u} + \vec{u}\dot\grad\vec{u} = -\frac{1}{\rho_0}\grad p + \frac{\rho'}{\rho_0}\vec{g} + \nu\grad^2 \vec{u} \label{eqn:momentum} \\
&\partial_t T + \vec{u}\dot\grad T + w \grad_{\rm{ad}} = \chi\grad^2 T' + \grad\dot[k \grad \overline{T}] + Q \label{eqn:temperature} \\
&\frac{\rho'}{\rho_0} = -|\alpha| T \label{eqn:boussinesq}
\end{align}
Where $\rho$ is the density, $T$ is the temperature, $\vec{g}$ is the gravitational acceleration, $\alpha$ is the coefficient of thermal expansion, $\nu$ and $\chi$ are the viscous and thermal diffusivity, $k$ is a radiative diffusivity, and $Q$ is a heating term.
We have baked in an assumption that $\alpha < 0$ to make sign conventions more straightforward after substituting Eqn.~\ref{eqn:boussinesq} into Eqn.~\ref{eqn:momentum}.

\section{Convective Penetration argument}
First some definitions:
\begin{enumerate}
\item $z=L_s$ is the the top of the convection zone according to the Schwarzschild criterion; it's the height where $\grad_{\rm{ad}} = \grad_{\rm{rad}}$.
\item $L_{\rm{cz}}$ is the top of the convection zone; it's roughly the top of the region where convection flattens $\grad \rightarrow \grad_{\rm{ad}}$.
We generally get $L_{\rm{cz}} > L_s$.
\item The penetration depth is $\delta_p = L_{\rm{cz}} - L_s$.
\item The flux carried by convection for $z < L_s$ is $F_{\rm{conv,cz}} = Q \delta_H$, where $Q$ is the magnitude of the internal heating and $\delta_H$ is the depth of the heating layer.
\item $\overline{w}$ is the vertical profile of the characteristic (vertical) convective velocity, which is a constant $w_{\rm{cz}}$ for $z \leq L_{\rm{cz}}$.
\item Similarly, $\overline{\delta T}$ is the vertical profile of the characteristic temperature perturbation.
\end{enumerate}

And some key assumptions.
\begin{enumerate}
\item Convection flattens $\grad \rightarrow \grad_{\rm{ad}}$ for $z \leq L_{\rm{cz}}$ (baked into our definitions).
\item We assume a system in thermal equilibrium, at least in (adiabatic) convection zone with $z \leq L_{\rm{cz}}$.
Therefore $F_{\rm{conv}}(z) = F_{\rm{tot}}(z) - F_{\rm{rad, ad}}(z)$, where $F$ is a flux and $F_{\rm{rad, ad}}$ is the radiative flux along the adiabatic gradient.
\item We assume $F_{\rm{conv}} = \overline{w T} \approx \overline{w}\overline{\delta T}$.
Combined with our previous assumption, we get
\begin{equation}
\overline{\delta T} \approx \frac{F_{\rm{tot}}(z) - F_{\rm{rad, ad}}(z)}{\overline{w}}.
\label{eqn:dT_assumption}
\end{equation}
\end{enumerate}
As we increase in height, the conductivity and thus $F_{\rm{rad, ad}}(z)$ also increases.
This means that $\overline{\delta T}$ has the opposite sign of $\overline{w}$ for $z > L_s$.

We presume that buoyancy breaking is the dominant mechanism which brings convective motions to a stop in this adiabatic layer.
In other words, we assume that, if we drop the nonlinear, pressure, and viscous terms from Eqn.~\ref{eqn:momentum}, we describe the dynamics reasonably well,
\begin{equation}
\frac{d \overline{w}}{dt} = \frac{\overline{\rho'}}{\rho_0}(-g\hat{z}) = \alpha g\, \overline{\delta T} = \alpha g \, \frac{F_{\rm{tot}}(z) - F_{\rm{rad, ad}}(z)}{\overline{w}}.
\end{equation}
If we multiply both sides by $\overline{w}$ and absorb it into the the derivative, then apply the chain rule with $d/dt = d/dz (dz/dt) = d/dz \overline{w}$, we retrieve,
\begin{equation}
\frac{1}{2} d\overline{w}^3 = \alpha g\, \left[F_{\rm{tot}}(z) - F_{\rm{rad, ad}}(z)\right]dz.
\end{equation}
We now replace $F_{\rm{rad, ad}}(z) = k_0(z) \grad_{\rm{ad}}$, and we assume that $k(z)$ instantaneously jumps from a low value $k_{\rm{cz}}$ to a high value $k_{\rm{rz}}$ at $z = L_s$.
Per this assumption, the convective flux is a negative constant for $z \geq L_s$.
We integrate from $z = L_s$ with $\overline{w} = w_{\rm{cz}}$ to $z = L_{s}$ with $\overline{w} = 0$, and get
\begin{equation}
-\frac{w_{\rm{cz}}^3}{2} = \alpha g\, F_{\rm{conv,p}}\,  \delta_p.
\end{equation}
By definition, the convective flux in the penetrative layer with $z > L_s$ is related to $\mathcal{P}$,
\begin{equation}
F_{\rm{conv,p}} = -\frac{F_{\rm{conv,cz}}}{\mathcal{P}},
\end{equation}
so we retrieve
\begin{equation}
\delta_p = \frac{w_{\rm{cz}}^3}{2\alpha g\, F_{\rm{conv, cz}}}  \mathcal{P},
\end{equation}
and the penetration depth scales with the cube of the velocity and linearly with $\mathcal{P}$.
This prediction does not exactly line up with the simulation results (this overpredicts the magnitude of $\delta_p$).
The important thing we need to determine is if it scales with $w_{\rm{cz}}$, $\mathcal{P}$, and $F_{\rm{conv,cz}}$ in the appropriate way.
So far, it seems to scale linearly with $\mathcal{P}$ at fixed other parameters.

Finally, we note that for an ideal gas with $P = \mathcal{R} \rho T$,
\begin{equation}
\alpha = \frac{\partial \ln\rho}{\partial T} = \frac{1}{\rho}\frac{\partial\rho}{\partial T} \approx - T^{-1}.
\end{equation}
If we had assumed that the convective flux had the form $\rho c_p w \delta T$ during Eqn.~\ref{eqn:dT_assumption}, we would have retrieved
\begin{equation}
\delta_p = \frac{\rho\, c_p\, w_{\rm{cz}}^3}{2\alpha g\, F_{\rm{conv, cz}}}  \mathcal{P}
\qquad \Rightarrow \qquad
\frac{\delta_p}{H_P} = \frac{\rho\, w_{\rm{cz}}^3}{2 F_{\rm{conv, cz}}}  \mathcal{P},
\end{equation}
with $H_P = c_P T / g$ the rough pressure scale height.

\section{Zahn argument}
Here we will work through a Boussinesq version of the argument in section 3.1 of \citet{zahn1991}.
In the adiabatic penetrative region, we know that the radiative flux is
\begin{equation}
F_{\rm{rad}} = -k \frac{\partial T_{\rm{ad}}}{\partial z}
\end{equation}
Further, we know that the convective flux is the difference between the total total and radiative flux,
\begin{equation}
F_{\rm{conv}} = F_{\rm{tot}} - F_{\rm{rad}}.
\end{equation}
Furthermore, the convective flux is
\begin{equation}
F_{\rm{conv}} = \overline{w T} = f \, W(z) \, \delta T(z),
\end{equation}
where $f$ is the filling factor of the flows that carry the flux in the penetration zone (upflows) and $W$ and $T$ give the vertical shape of typical velocity and temperature perturbations.

We next assume that the temperature and velocity are highly correlated in the overshooting convective motions (upflows).
This allows us to describe the horizontal nature of all fields in upflows using a function $h(x,y)$ so that
\begin{equation}
w(x,y,z) = W(z) h(x,y)
\qquad
T'(x,y,z) = \delta T(z) h(x,y),
\end{equation}
where $T'$ are the temperature fluctuations away from the mean.
By continuity and by our above definition of the convective flux, we have
\begin{equation}
\overline{h} = 0,
\qquad
\overline{h^2} = f.
\end{equation}



\bibliography{biblio.bib}
\end{document} 
