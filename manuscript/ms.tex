% Preamble
\documentclass{aastex631}
\usepackage{natbib}
\usepackage{latexsym}
\usepackage{graphicx}
\usepackage{epsfig}
\usepackage{amssymb}
\usepackage{amsmath}
\usepackage{epstopdf}
\usepackage{hyperref}

%%%% Custom commands
\newcommand{\gradrad}{\ensuremath{\nabla_{\rm{rad}}}}
\newcommand{\gradad}{\ensuremath{\nabla_{\rm{ad}}}}
\newcommand{\Ftot}{\ensuremath{F_{\rm{tot}}}}
\newcommand{\Frad}{\ensuremath{F_{\rm{rad}}}}
\newcommand{\Fconv}{\ensuremath{F_{\rm{conv}}}}
\newcommand{\mP}{\ensuremath{\mathcal{P}}}
\newcommand{\mR}{\ensuremath{\mathcal{R}}}
\newcommand{\mS}{\ensuremath{\mathcal{S}}}
\newcommand\Pran{\ensuremath{\mathrm{Pr}}}

\newcommand{\pd}[1]{\partial_{#1}}
\renewcommand{\vec}[1]{\boldsymbol{#1}}
\newcommand{\M}[1]{\mathbf{#1}}
\renewcommand{\dot}{\vec{\cdot}}
\newcommand{\grad}{\vec{\nabla}}
\newcommand{\cross}{\vec{\times}}
\newcommand{\laplacian}{\nabla^2}

%%%% Journal preamble
\received{}
\revised{}
\accepted{}
\published{}
\submitjournal{ApJ}

\shorttitle{SHORT TITLE}
\shortauthors{Anders et al}


\begin{document}

%%%% Title and Abstract
\title{This is a title}
\author[0000-0002-3433-4733]{Evan H. Anders}
\affiliation{CIERA, Northwestern University}
\author[0000-0001-5048-9973]{Adam S. Jermyn}
\affiliation{CCA, Flatiron Institute}
\author{Daniel Lecoanet}
\affiliation{ESAM \& CIERA, Northwestern University}
\author[0000-0001-8935-219X]{Benjamin P. Brown}
\affiliation{APS Dept and LASP, University of Colorado, Boulder}

\correspondingauthor{Evan H. Anders}
\email{evan.anders@northwestern.edu}

\begin{abstract}
Blah Blah short description
\end{abstract}
\keywords{UAT keywords}



%%%% Body of paper
\section{Introduction}
\label{sec:introduction}
Convection is a crucial heat transport mechanism over some fraction of the stellar radius for all stars at some point in the stellar lifetime [CITE].
These motions drive the magnetic dynamo of the Sun and other stars [CITE], leading to the host of emergent phenomena known as stellar activity.
Furthermore, convective motions impinge upon nearby stable layers, exciting gravity waves [CITE].
Convection is also responsible for mixing chemical compositions, which becomes increasingly important in the cores of evolved stars [CITE].
A complete and nuanced understanding is therefore crucial for understanding stellar structure, evolution, and observations.

One particular aspect of stellar convection which remains poorly understood even after decades of study is the class of mechanisms generally referred to as ``convective overshoot.''
Useful parameterizations of the way in which convective motions extend beyond the nominally Schwarzschild- or Ledoux- stable boundaries of the convective zone have historically been elusive.
Improved models of this ``overshoot'' could help to resolve many discrepancies between observations and theoretical model.
In the Sun and solar-type stars, better models of convective boundaries could help solve the mystery of low abundances of Li at the surface of solar-type stars \citep{pinsonneault1997, dumont_etal_2021}, the ``solar modeling problem'' \citep{basu_antia_2004, bahcall_etal_2005, zhang_li_2012, vinyoles_etal_2017, asplund_etal_2021} and problems in helioseismic profiles near the base of the convection zone \citep{christensen-dalsgaard_etal_2011}.
There is also ample evidence that we do not understand the nature of convective mixing at the boundary of core convection zones \citep{claret_torres_2018, jermyn_etal_2018, viani_basu_2020, martinet_etal_2021, pedersen_etal_2021} which could have profound implications for the post-main sequence evolution and remnants of massive stars \citet{farmer_etal_2019, higgins_vink_2020}.
In order to ensure that models can be evolved on fast (human) timescales, 1D stellar evolution codes rely on simple parameterizations of convective overshoot and mixing beyond convective boundaries \citep{shaviv_salpeter_1973, maeder1975, herwig2000, paxton_etal_2011, paxton_etal_2013, paxton_etal_2018, paxton_etal_2019}.
While some preliminary work has been done to couple 3D dynamical convective simulations with 1D stellar evolution codes [CITE], these calculations are currently prohibitively expensive to perform e.g., at every timestep in a stellar evolution calculation.
In short, an improved theoretical understanding of the behavior of convective boundaries which can inform easy-to-calculate parameterizations is essential.

Convective overshoot and penetration have been studied in laboratory experiments and numerical simulations for decades, and has been reviewed by many authors \citep{marcus_etal_1983, zahn1991, browning_etal_2004, rogers_etal_2006, viallet_etal_2015, korre_etal_2019}.
%Early experiments of convective penetration of water in the laboratory \citep{deardorff_etal_1969} or simple numerical plumes \citep{schmitt_etal_1984} have given way to a wide array of dynamical experiments in recent years.
A slew of simulations in Cartesian \citep{musman1968, moore_weiss_1973, hurlburt_etal_1986, hurlburt_etal_1994, singh_etal_1995, saikia_etal_2000, brummell_etal_2002, rogers_glatzmaier_2005, kapyla_etal_2007, tian_etal_2009, kitiashvili_etal_2016, lecoanet_etal_2016, kapyla_etal_2017, couston_etal_2017, toppaladoddi_wettlaufer_2018, kapyla2019, cai2020} and spherical \citep{browning_etal_2004, rogers_etal_2006, brun_etal_2017, pratt_etal_2017, dietrich_wicht_2018, higl_etal_2021} geometry have been studied.
Despite this lengthy list of experiments, no consensus model of convective overshoot or penetration has emerged.
Throughout the remainder of this work, we will use terminology from this hydrodynamical literature.
``Convective penetration'' refers to convective motions which extend beyond the nominal Schwarzschild boundary of the convection zone \emph{and} flatten the temperature gradient towards the adiabatic.
``Convective overshoot'' refers to the motions that extend beyond the convective boundary but do not modify the thermal structure.
Our main focus in this paper will be on convective penetration.

\citet{zahn1991} theorized that convective penetration should depend only on how steeply the radiative temperature gradient varies at the convective boundary.
Some simulations \citep{hurlburt_etal_1994, rogers_etal_2006} have shown at least partial agreement with this theory.
A semianalytic model of solar overshoot \citep{rempel2004} also agreed with the early ideas of Zahn. 
Furthermore, some select simulations have found hints that convective overshoot or penetration may be sensitive the magnitude of the flux in some way \citep{singh_etal_1998, hotta2017, kapyla2019}.
These results suggest that the gradients of fluxes near convective boundaries deserve further examination.

In this work, we design two numerical experiments to test the theory of \citet{zahn1991}.
We use a modified incompressible, Boussinesq model to study the simplest possible system, and re-derive his theory in our simplified limit.
The results of our simulations are in full agreement with Zahn's theory.
\begin{quote}
\emph{
Specifically, we find that the depth of convective penetration depends on the gradient of the radiative flux near the convective boundary.
}
\end{quote}
Thus, the penetration depth can be approximated so long as the radiative conductivity, or likewise the opacity, is known at the convective boundary.

We present these findings as follows.
In Sec.~\ref{sec:theory}, we describe our modified Boussinesq equations, re-derive the theory of \citet{zahn1991}, and retrieve predictions for our two experimental designs from that theory.
In Sec.~\ref{sec:simulation_details}, we describe our simulation setup and parameters.
In Sec.~\ref{sec:results}, we present the results of these simulations, with a particular focus on the depth of the penetrative regions.
In Sec.~\ref{sec:solar_model}, we create and discuss a solar MESA model which uses this theory to determine the bottom of the solar convection zone.
Finally, we discuss how future simulations can put finer constraints on this theory in Sec.~\ref{sec:discussion}.

\section{Theory}
\label{sec:theory}
Throughout this work, we will utilize a modified version of the Boussinesq equations of motion, similar to the model derived by \citet{spiegel_veronis_1960} and utilized by e.g., \citet{korre_etal_2019}.
In dimensional form,
\begin{align}
&\grad\dot\vec{u} = 0 
\label{eqn:incompressible} \\
&\partial_t \vec{u} + \vec{u}\dot\grad\vec{u} = -\frac{1}{\rho_0}\grad p + \frac{\rho'}{\rho_0}\vec{g} + \nu\grad^2 \vec{u} 
\label{eqn:momentum} \\
&\partial_t T + \vec{u}\dot\grad T + w \gradad = \chi\grad^2 T' + \grad\dot[k \grad \overline{T}] + Q
\label{eqn:temperature} \\
&\frac{\rho'}{\rho_0} = -|\alpha| T.
\label{eqn:boussinesq}
\end{align}
In this model, $\rho_0$ is a (constant) background density and $\rho'$ are fluctuations which act only in the buoyancy force and varies linearly with the temperature $T$ according to the coefficient of thermal expansion, $\alpha = \partial\ln\rho / \partial T$.
Furthermore, $\vec{u}$ is the velocity vector, $\nu$ and $\chi$ are respectively the viscous and thermal diffusivity, $Q$ is a bulk internal heating term [CITE], and $\gradad$ is the adiabatic temperature gradient (we define $\gradad$ as a positive value to align with stellar structure conventions; this means marginal stability is achieved when $\partial_z T = -\gradad$).
We modify the model of \citet{spiegel_veronis_1960} to allow the mean temperature profile $\overline{T}$ to carry a radiative flux $\Frad = -k \grad \overline{T}$, where $k$ is a radiative diffusivity which can vary with height.
We assume that the classical thermal diffusion term $\chi \grad^2 T'$  only acts on the fluctuations away from the mean temperature profile, $T' \equiv T - \overline{T}$.




\section{Simulation Details}
\label{sec:simulation_details}
\begin{align}
&\grad\dot\vec{u} = 0 \\
&\partial_t \vec{u} + \vec{u}\dot\grad\vec{u} = -\grad \varpi + T \hat{z} + \mR^{-1}\grad^2 \vec{u} \\
&\partial_t T + \vec{u}\dot\grad T + w \grad_{\rm{ad}} = (\Pran\mR)^{-1}\grad^2 T' + \grad\dot[k \grad \overline{T}] + Q
\end{align}

\section{Results}
\label{sec:results}

\section{A modified solar model}
\label{sec:solar_model}

\section{Discussion}
\label{sec:discussion}

\begin{acknowledgments}
We'd like to thank
\end{acknowledgments}


\appendix

\section{Accelerated Evolution}

\section{Table of simulation parameters}





\bibliographystyle{aasjournal}
\bibliography{biblio}
\end{document}
